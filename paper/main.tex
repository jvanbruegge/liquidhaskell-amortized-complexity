%\documentclass[sigplan,screen]{acmart}
\documentclass[sigplan,screen,review,anonymous]{acmart}

\usepackage{xcolor}
\definecolor{codegreen}{rgb}{0,0.6,0}

\usepackage{cancel}

\usepackage{listings}
\lstset{%
    language=Haskell,
    columns=fullflexible,
    basicstyle=\ttfamily,
    commentstyle=\color{codegreen}
}

% shorthand for big O
\renewcommand\O[1]{$\mathcal{O}(#1)$}

%% Rights management information.  This information is sent to you
%% when you complete the rights form.  These commands have SAMPLE
%% values in them; it is your responsibility as an author to replace
%% the commands and values with those provided to you when you
%% complete the rights form.
\setcopyright{acmlicensed}
\copyrightyear{2021}
\acmYear{2021}
\acmDOI{tbd}

%% These commands are for a PROCEEDINGS abstract or paper.
\acmConference[Haskell '21]{the 14th ACM SIGPLAN International Haskell Symposium}{August 26--27, 2021}{Virtual Event}
%\acmBooktitle{Woodstock '18: ACM Symposium on Neural Gaze Detection,
%  June 03--05, 2018, Woodstock, NY}
%\acmPrice{15.00}
%\acmISBN{978-1-4503-XXXX-X/18/06}


%%
%% Submission ID.
%% Use this when submitting an article to a sponsored event. You'll
%% receive a unique submission ID from the organizers
%% of the event, and this ID should be used as the parameter to this command.
%%\acmSubmissionID{123-A56-BU3}

%%
%% end of the preamble, start of the body of the document source.
\begin{document}

%%
%% The "title" command has an optional parameter,
%% allowing the author to define a "short title" to be used in page headers.
\title{Liquid Amortization - Proving amortized complexity with LiquidHaskell (Functional Pearl)}

%%
%% The "author" command and its associated commands are used to define
%% the authors and their affiliations.
%% Of note is the shared affiliation of the first two authors, and the
%% "authornote" and "authornotemark" commands
%% used to denote shared contribution to the research.
\author{Jan van Brügge}
\orcid{0000-0003-1560-7326}
\affiliation{%
  \institution{Technical University of Munich}
  \streetaddress{Arcisstraße 21}
  \city{Munich}
  \state{Bavaria}
  \country{Germany}
  \postcode{80333}
}
\email{jan.van-bruegge@tum.de}

%%
%% By default, the full list of authors will be used in the page
%% headers. Often, this list is too long, and will overlap
%% other information printed in the page headers. This command allows
%% the author to define a more concise list
%% of authors' names for this purpose.
%\renewcommand{\shortauthors}{Trovato and Tobin, et al.}

%%
%% The abstract is a short summary of the work to be presented in the
%% article.
\begin{abstract}
Proofs of complexity of production code are usually done in external theorem provers. This makes it necessary to translate the data structures to be verified from the production language into the language of the prover. Such a translation step could introduce errors, for example due to a mismatch in features between the two languages. We show how to prove amortized complexity of data structures directly in Haskell using LiquidHaskell~\cite{liquidhaskell}. Besides skipping the translation step, our approach also provides a didactic advantage as learners do not have to learn an additional language for their proofs and can focus on the new concepts only. We do not assume prior knowledge of amortized complexity as we explain the concepts and apply them in our first case study, a simple stack with multipop. The practicality of our approach in a production setting is demonstrated by our second case study: the finger tree. It powers \texttt{Data.Seq} from the \textit{containers} package~\cite{containers} which is widely depended on. Finally we discuss the current limitations of LiquidHaskell in this area and how it might be possible to improve the situation.
\end{abstract}

%%
%% The code below is generated by the tool at http://dl.acm.org/ccs.cfm.
%% Please copy and paste the code instead of the example below.
%%
\begin{CCSXML}
<ccs2012>
   <concept>
       <concept_id>10003752.10003809.10010031</concept_id>
       <concept_desc>Theory of computation~Data structures design and analysis</concept_desc>
       <concept_significance>500</concept_significance>
       </concept>
   <concept>
       <concept_id>10011007.10011006.10011008.10011009.10011012</concept_id>
       <concept_desc>Software and its engineering~Functional languages</concept_desc>
       <concept_significance>500</concept_significance>
       </concept>
   <concept>
       <concept_id>10011007.10010940.10010992.10010998.10010999</concept_id>
       <concept_desc>Software and its engineering~Software verification</concept_desc>
       <concept_significance>500</concept_significance>
       </concept>
 </ccs2012>
\end{CCSXML}

\ccsdesc[500]{Theory of computation~Data structures design and analysis}
\ccsdesc[500]{Software and its engineering~Functional languages}
\ccsdesc[500]{Software and its engineering~Software verification}

%%
%% Keywords. The author(s) should pick words that accurately describe
%% the work being presented. Separate the keywords with commas.
\keywords{datastructures, LiquidHaskell, amortized complexity, finger trees, theorem proving, functional pearl}

%%
%% This command processes the author and affiliation and title
%% information and builds the first part of the formatted document.
\maketitle

\section{Introduction}

Formally proving certain properties of production code is becoming more and more common. For safety critical code that is not allowed to "go wrong" as well as code that is fundamental and reused very often like core libraries, bugs can have a widespread impact. Proving that the code behaves according to the specification can provide an extra layer of confidence. Fundamental data structures are one of the most important parts of a programming language ecosystem and a lot of production code depends on them. In the Haskell ecosystem, this job is done by the \textit{containers} package which has over 5300 reverse dependencies\footnote{reverse dependecies are other packages that depend on \textit{containers}} on Hackage~\cite{hackage}, the main repository for Haskell packages. A lot of code depends on the assumptions that the data structures are fast and adhere to computational complexities stated in their documentation. Thus, proving those assumptions correct benefits a huge number of people.

Usually these proofs are done in an external theorem prover (e.g. Coq~\cite{complexity_coq} or Isabelle/HOL~\cite{complexity_isabelle}), requiring an explicit translation step from the production language to the language the theorem prover uses. This translation step has a few disadvantages. First, the semantics of the target language could differ from the semantics of the production language. This is especially true for Haskell with its pervasive laziness, compared to most theorem provers that operate on a strict language. Second, such a translation step could introduce subtle differences or bugs, either by accident when doing a manual translation or through a bug when an automated translation tool is used. Lastly, the target language may not support certain features present in the source language (e.g. polymorphic recursion in data types, see section~\ref{sec:fingertree}). In those cases, the missing features have to be emulated in the target language, again opening the door for small and subtle errors. Because the code that is actually used in the formal proofs differs from the source code that is running in production, it is often hard to reconnect the proofs back to the original production code.

A different approach comes in the form of LiquidHaskell~\cite{liquidhaskell}. It is a refinement type~\cite{refinement_types} checker plugin\footnote{before 2020, LiquidHaskell was implemented as a standalone executable, it now being a GHC plugin~\cite{lh_plugin} drastically improves usability} for GHC, the main Haskell compiler used in industry. It allows specifing additional constraints on types, e.g. "integers greater than zero" instead of just "integers". These constraints are propagated through the code and checked by an external SMT solver. With a handful of proof combinators defined by LiquidHaskell, it is possible to use Haskell directly as a theorem prover, enabling intuitive equational reasoning within the language~\cite{tpfa}. This avoids any translation steps and makes it easy to keep implementation and its respective proof in sync, as any failure to do so will result in a compiler error. Another benefit is that the reader or author of a proof does not need to learn an additional language, the production language is the proof language.

In this paper, we:
\begin{itemize}
\item{Assume no prior knowledge of amortized complexity as we recap the theoretical background (section~\ref{sec:complexity})}
\item{Show the reasoning infrastructure that is needed to prove amortized complexity using LiquidHaskell (sections~\ref{sec:liquidhaskell} and~\ref{sec:stack})}
\item{Apply this technique to \textit{finger trees}, the data structure behind \texttt{Data.Seq} from \textit{containers} (see our second case study in section~\ref{sec:fingertree})}
\item{Discuss the usefulness of LiquidHaskell for such proofs compared with other theorem provers (sections~\ref{sec:limitations} and~\ref{sec:conclusion})}
\end{itemize}

\section{Amortized complexity}\label{sec:complexity}

It is common practise to check how long an operation on a data structure takes in relation to the number of elements in the data structure. As the performance can vary wildly for the same operation depending on the state of data structure, usually the worst case for each operation is used as a metric and specified using \textit{Landau notation} ("big O"). For example, a linear search takes $1$ unit of time if the searched element is at the front, while it takes $n$ units of time if it is the last element. Thus we would say that a linear search is in the complexity class of \O{n} because it will not take longer than $c \cdot n$ units of time for some fixed constant $c$.

But often this worst case is too pessimistic. When an operation changes the data structure, for example by removing elements, consecutive operations might be cheaper. Or to use an expensive operation it might be necessary to have a certain amount of cheap operations. In such cases one uses \textit{amortized complexity}, which does not only look at a single operation in isolation, but at chains of operations.

As a running example we will use a stack with \textit{push} and \textit{multipop} operations. The former allocates a new element and a reference to the rest of the stack and thus takes a constant amount of time. The latter removes the first $k$ elements from the stack. In the worst case, it removes all the elements, making \texttt{multipop} \O{n}. One can already see that this is too pessimistic, because after removing all elements from the stack, the next \texttt{multipop} will do nothing. In fact, to be able to remove an element, it has had to be pushed to the stack at some point, so we can spread out the cost of \texttt{multipop} over those pushes. There are multiple techniques to prove amortized complexity, we present the two most popular ones.

\subsection{Banker's method}\label{sec:banker}

The banker's method works by having a bank account that can store time units. Operations have to pay their actual runtime cost and may additionally deposit into or withdraw from the account. The sum of the actual cost and the deposit/withdrawal is the \textit{amortized cost} of the operation. The idea is that cheap operations pay extra time units, making their amortized cost higher than their actual cost. The goal is to keep the difference between the two costs to a constant factor because that means the operation still retains their (cheap) complexity class. Expensive operations on the other hand can use the stored time to pay for their actual cost, hopefully putting them in a cheaper complexity class. In our example a \texttt{push} has to pay one actual time unit and it will also pay one extra time unit into the bank account. The difference between amortized cost and actual cost is constant, so \texttt{push} is still in \O{1}. Because for every element pushed we deposit one extra time unit into the bank, the account always contains as many time units as there are elements on the stack. As \texttt{multipop} can not remove more elements than there are on the stack, withdrawing $k$ time units to pay for the actual cost will never make the account become negative. Because withdrawing from the bank account is enough to pay for the complete actual cost of \texttt{multipop} regardless of the choice of $k$, its \textit{amortized} runtime is in \O{1}.

\subsection{Physicist's method}\label{sec:physicist}

While the banker's method is easy to visualize, it is hard to use in a formal proof because the bank account acts as state between the operations. A simpler method is the physicist's method, which does not need such an account. The idea is to define a potential $\Phi$ ("phi") for the data structure. This potential is defined to be $0$ for an empty data structure and $\ge 0$ for all other states. Intuitively, the potential represents how much time needs to be saved up to pay for expensive operations. We then define the amortized cost of an operation as $c + \Phi(h') - \Phi(h)$ where $c$ is the actual time cost of the operation, $h$ is the data structure before the operation and $h'$ is the data structure afterwards. Looking at a chain of operations $1, 2, ..., n$, the sum of the amortized time is an overestimation of the actual cost by $\Phi(h_n)$ and thus an upper bound on the actual cost (see equation~\ref{eq:sum}, remember that by definition $\Phi(h_0) = 0$).

\begin{gather}
\nonumber (c_1 + \cancel{\Phi(h_1)} - \Phi(h_0)) + (c_2 + \cancel{\Phi(h_2)} - \cancel{\Phi(h_1)}) \\
\nonumber    + ... + (c_n + \Phi(h_n) - \cancel{\Phi(h_{n - 1})}) = \\
  c_1 + c_2 + ... + c_n + \Phi(h_n) - \Phi(h_0) \label{eq:sum}
\end{gather}

For our stack we define the potential to be the height of the stack. This fulfils the requirements as the empty stack has a potential of zero and all other states have a potential greater or equal to zero. For the actual proof of this example see the first case study in section~\ref{sec:stack}.

\section{LiquidHaskell for theorem proving}\label{sec:liquidhaskell}

As described by Vazou et. al.~\cite{tpfa}, one can use LiquidHaskell to do complex equational reasoning that is machine checked. It uses refinement types~\cite{refinement_types} that add additional constraints on top of normal Haskell data types which are checked by an external SMT solver. By default LiquidHaskell uses the z3~\cite{z3} solver, but it also supports cvc4~\cite{cvc4}. For example, we can declare the type of integers greater or equal to zero using a type synonym:

\begin{lstlisting}
{-@ type Nat = { x:Int | x >= 0 } @-}
\end{lstlisting}

LiquidHaskell annotations are multiline Haskell comments with "at" signs, such that the code still compiles even without LiquidHaskell. It also allows the user to name argument types, so they can be referred to later. At the time of writing one has to copy the Haskell type signature into a LiquidHaskell comment and add the refinements there. A future version that is integrated deeper into the Haskell compiler might not need this code duplication.

In LiquidHaskell, all functions need to terminate. Usually a suitable termination metric is automatically deduced, but in some more complex cases, we can also manually supply this metric (see section~\ref{sec:append} for an example). With the termination, LiquidHaskell is able to generate an induction principle for the function that can be used in a proof. Because LiquidHaskell is just Haskell, this induction amounts to a recursive call to the proof and is also used when defining ordinary Haskell functions with refinement on their types. For example the fact that the length of a list is never negative:

\begin{lstlisting}
{-@ measure length @-}
{-@ length :: [a] -> Nat @-}
length :: [a] -> Int
length [] = 0 -- trivial, 0 >= 0
length (x:xs) = 1 + length xs
    -- by induction: length xs >= 0
    -- thus 1 + length xs >= 0
\end{lstlisting}

To use functions in the refinements, one has to \textit{reflect}~\cite{reflection} them to the logic level. For a limited class of functions called \textit{measures}, this can be done while retaining full automatic reasoning. A measure is a function that takes only one argument which is an algebraic data type, has one equation per constructor of this data type and which only uses arithmetic functions and other measures in their right hand sides. Other function still can be used in refinements when annotated with \texttt{reflect}, but the SMT solver is not able to check those refinement automatically. The user always has to provide the proofs themselves.

To do so, LiquidHaskell provides a set of proof combinators and a \texttt{Proof} type. This type is just a type alias for the normal Haskell unit type. The reasoning comes from the \texttt{===} operator that asserts that both sides of the equality are truly equal. For a simple example we prove that the length of \texttt{append} applied to two lists is the sum of the lengths of the two lists. Because \texttt{append} takes more than one argument, it is not a measure and thus we need to do the proof manually.

First, we need to do a case distinction on the first argument. This is done by pattern matching on it like in a normal Haskell function. The rest of the proof is done by applying the definitions of the used functions, arithmetic equalities and induction:

\begin{lstlisting}
{-@ reflect append @-}
append :: [a] -> [a] -> [a]
append [] ys = ys
append (x:xs) ys = x:(append xs ys)

{-@ lengthP :: xs:[a] -> ys:[a] ->
  { length (append xs ys) ==
      length xs + length ys }
@-}
lengthP :: [a] -> [a] -> Proof

-- base case
lengthP [] ys = length (append [] ys)
  -- Use definition of append [] _
    === length ys
    === 0 + length ys
  -- use definition of length []
    === length [] + length ys
    *** QED

-- recursive case
lengthP (x:xs) ys =
    length (append (x:xs) ys)
  -- Use definition of append
    === length (x:(append xs ys))
  -- Use definition of length
    === 1 + length (append xs ys)
  -- Use induction with the ? operator
    ? lengthP xs ys
    === 1 + length xs + length ys
  -- Use definition of length
    === length (x:xs) + length ys
    *** QED
\end{lstlisting}

The triple star operator (\texttt{***}) and the \texttt{QED} type at the end are just there to cast our proof to Haskell's unit type and improve readability.

\section{Case studies}

\subsection{Stack with Multipop}\label{sec:stack}

To continue with the example from section~\ref{sec:complexity}, we prove that $multipop$ has an amortized complexity of \O{1}. This data structure and its operations are defined like the following in Haskell. We already add some refinements to assert that $k$ is not negative. Note that normally, \texttt{multipop} would return the removed elements in addition to the new stack, but as this is irrelevant to the proof and only makes it more verbose we omit this in our definition.

\begin{lstlisting}
data Stack a = Empty | Elem a (Stack a)

{-@ reflect push @-}
push :: a -> Stack a -> Stack a
push x s = Elem x s

{-@ reflect multipop @-}
{-@ multipop :: Nat -> Stack a -> Stack a @-}
multipop :: Int -> Stack a -> Stack a
multipop _ Empty = Empty
multipop 0 s = s
multipop n (Elem _ s) = multipop (n - 1) s
\end{lstlisting}

To formally prove the amortized complexity with the physicist's method, we first need to define a potential $\Phi$ and a timing function for each of our operations that gives us the actual time cost of that operation. As seen in section~\ref{sec:physicist}, we use the height of the stack as our potential. Thus we need to define a function that calculates the height of a stack. To define the timing functions, we copy the code for the operation, replace constant time steps with $1$ and adjust the recursion:

\begin{lstlisting}
{-@ reflect phi @-}
{-@ phi :: Stack a -> Nat @-}
phi :: Stack a -> Int
phi Empty = 0
phi (Elem _ s) = 1 + phi s

{-@ reflect pushT @-}
{-@ pushT :: a -> Stack a -> Nat @-}
pushT :: a -> Stack a -> Int
pushT _ _ = 1

{-@ reflect multipopT @-}
{-@ multipopT :: Nat -> Stack a -> Nat @-}
multipopT :: Int -> Stack a -> Int
multipopT _ Empty = 1
multipopT 0 _ = 1
multipopT n (Elem _ s) = 1 + multipopT (n - 1) s
\end{lstlisting}

First we will prove the complexity of \texttt{push}. While the amortized cost is in the same complexity class as the actual cost, we need to make sure that our potential function does not cause those two to diverge. LiquidHaskell does not allow quantifiers in refinement, because for functions other than measures, instanciating quantifiers in the SMT solver has unpredictable performance~\cite{lh_quantifiers}. This means we have to guess an upper limit for our amortized runtime. As seen earlier in section~\ref{sec:banker}, we can pay two time units for a push to amortize the pops, so we will use $2$ as our guess:

\begin{lstlisting}
import Language.Haskell.Liquid.ProofCombinators

{-@ pushP :: x:a -> s:Stack a ->
  { pushT x s + phi (push x s) - phi s <= 2 }
@-}
pushP :: a -> Stack a -> Proof
pushP x s = pushT x s + phi (push x s) - phi s
-- Use definition of push and pushT
  === 1 + phi (Elem x s) - phi s
-- Use definition of phi (Elem _ _)
  === 1 + (1 + phi s) - phi s
-- phi s gets canceled, 1 + 1 <= 2
  *** QED
\end{lstlisting}

To prove the complexity of \texttt{multipop}, we will need to distinquish between the two base cases and the recursive case. With LiquidHaskell this is just a normal pattern match. Leaving the recursive case undefined for now, proving the base cases is very similar to \texttt{push}:

\begin{lstlisting}
{-@ multipopP :: k:Nat -> s:Stack a ->
  { multipopT k s + phi (multipop k s)
      - phi s <= 2 }
@-}
multipopP :: Int -> Stack a -> Proof
multipopP 0 s = multipopT 0 s
    + phi (multipop 0 s) - phi s
-- Use definition of multipop/T 0 _
  === 1 + phi s - phi s
-- phi s cancels, 1 <= 2
  *** QED
multipopP k Empty = multipopT k Empty
    + phi (multipop k Empty) - phi Empty
-- Use definition of multipop/T _ Empty
  === 1 + phi Empty - phi Empty
-- phi Empty cancels, 1 <= 2
  *** QED
\end{lstlisting}

To avoid cluttering the proof of the recursive case, we use an \textit{as pattern} to alias \texttt{Elem x s} to \texttt{xs}. The first steps are again applying the definitions of the used functions and simplifying the result. In the end we use the \texttt{?} operator to add the induction hypothesis to the proof step and thus complete the proof:

\begin{lstlisting}
multipopP k xs@(Elem x s) = multipopT k xs
    + phi (multipop k xs) - phi xs
-- Use definition of multipopT for recursive case
  === (1 + multipopT (k - 1) s)
    + phi (multipop k xs) - phi xs
-- Use definition of multipop for recursive case
  === (1 + multipopT (k - 1) s)
    + phi (multipop (k - 1) s) - phi xs
-- Use definition of phi (Elem _ _)
  === (1 + multipopT (k - 1) s)
    + phi (multipop (k - 1) s) - (1 + phi s)
-- Remove parentheses
  === 1 + multipopT (k - 1) s
    + phi (multipop (k - 1) s) - 1 - phi s
-- 1s cancel out
  === multipopT (k - 1) s
    + phi (multipop (k - 1) s) - phi s
-- use induction hypothesis
  ? multipopP (k - 1) s
  *** QED
\end{lstlisting}

\subsection{Finger Tree}\label{sec:fingertree}

To show that this approach scales to more complicated data structures that are widely used in the industry, we will prove the complexity of finger trees, the data structure behind \texttt{Data.Seq} from \textit{containers}~\cite{containers}. It was originally described by Hinze and Paterson~\cite{fingertrees}, the implementation from \textit{containers} closely follows this description. The following definition is taken directly from \textit{containers}, but slightly modified to remove the size parameter. LiquidHaskell caused problems in the proofs when using type classes because at the time of writing it is not possible to replace a call to a polymorphic function with the implementation in its instance. None of the function we use in the proofs use this size parameter, so we omit it. We will discuss such limitations more in depth in section~\ref{sec:limitations}.

\begin{lstlisting}
data FingerTree a = Empty
  | Single a
  | Deep (Digit a) (FingerTree (Node a)) (Digit a)

data Node a = Node2 a a | Node3 a a a

data Digit a = One a
  | Two a a
  | Three a a a
  | Four a a a a
\end{lstlisting}

The finger tree has between one and four \textit{digits} on either side to enable cheap \texttt{cons} and \texttt{snoc}. As shown by Claessen~\cite{fingertrees_new}, this would also work with a maximum of just three digits, but we will follow the definition from \textit{containers}. The most interesting property of the data structure is the polymorphic recursion. It makes the data structure type safe as it guarantees equal nesting of \texttt{Node} types on both sides of the spine. It is also what makes it difficult to verify in other languages, because for example the both Isabelle/HOL~\cite{isabelle} and the Lean~\cite{lean} theorem provers do not directly support polymorphic recursion. In these languages, it is necessary to define the type in such a way that it would allow uneven nesting of \texttt{Node}s and then "carve out" the subset of valid finger trees. Other provers like Coq~\cite{coq} support polymorphic recursion, allowing a more direct one to one translation.

\subsubsection{Cons and Snoc}

Because the finger tree is a symmetric tree, \texttt{cons} and \texttt{snoc} are very similar. For this reason we show an explicit proof only for \texttt{cons} and provide the proof for \texttt{snoc} in the appendix. First we again copy the definition of \texttt{cons} from the \textit{containers} package, adjusting it slightly to remove the size information that would require the type class. In order to later use the function in a proof we add a \texttt{reflect} pragma.

\begin{lstlisting}
{-@ reflect cons @-}
cons :: a -> FingerTree a -> FingerTree a
cons a EmptyT = Single a
cons a (Single b) =
  Deep (One a) EmptyT (One b)
cons a (Deep (One b) m sf) =
  Deep (Two a b) m sf
cons a (Deep (Two b c) m sf) =
  Deep (Three a b c) m sf
cons a (Deep (Three b c d) m sf) =
  Deep (Four a b c d) m sf
cons a (Deep (Four b c d e) m sf) =
  Deep (Two a b) (cons (Node3 c d e) m) sf
\end{lstlisting}

The only complex case is if we already have four digits at the front, as in all other cases we can simply grow the digit in size. In the complex case, we put the new data in front and push the "overflow" into the recursion. The next step is to find a valid potential function $\Phi$. For this, we check which states of the data structure make \texttt{cons} expensive. As evident by the definition above, this is the case when the digit in the front is already full. So we can define the \textit{danger} of a digit as every state that will make the operation expensive. To also support \texttt{uncons}, the inverse operation, we not only mark a full digits as dangerous, but also one that has only one element inside (as this would make \texttt{uncons} go into the recursion). We then define the potential to be the sum of the danger in the tree.

\begin{lstlisting}
{-@ reflect dang @-}
{-@ dang :: Digit a -> Nat @-}
dang :: Digit a -> Int
dang One{} = 1
dang Two{} = 0
dang Three{} = 0
dang Four{} = 1

{-@ reflect phi @-}
{-@ phi :: FingerTree a -> Nat @-}
phi :: FingerTree a -> Int
phi EmptyT = 0
phi Single{} = 0
phi (Deep pr m sf) =
  dang pr + phi m + dang sf
\end{lstlisting}

The timing function for \texttt{cons} is similar to the timing functions for the stack just a copy of the original function where all non-recursive cases return $1$.

\begin{lstlisting}
{-@ reflect consT @-}
{-@ consT :: a -> FingerTree a -> Nat @-}
consT :: a -> FingerTree a -> Int
consT _ EmptyT = 1
consT _ Single{} = 1
consT _ (Deep One{} _ _) = 1
consT _ (Deep Two{} _ _) = 1
consT _ (Deep Three{} _ _) = 1
consT _ (Deep (Four _ c d e) m _) =
  1 + consT (Node3 c d e) m
\end{lstlisting}

With all the required preliminaries defined, we start proving the amortized complexity. Just like in the stack example (see section~\ref{sec:stack}) we need to guess a constant that is an upper bound of our amortized run time. Because we do not have a simple mental model like with the stack, we will delay this choice until we have proven the base cases. We will also use \textit{as patterns} again to avoid repetition.

\begin{lstlisting}
{-@ consP :: x:a -> t:FingerTree a ->
  { consT x t + phi (cons x t) - phi t <= c }
@-}
consP :: a -> FingerTree a -> Proof
consP x t@EmptyT =
    consT x t + phi (cons x t) - phi t
-- Use definition of cons/T _ EmptyT
  === 1 + phi (Single x) - phi EmptyT
-- Use definition of phi
  === 1 + 0 - 0
-- 1 must be <= c, so c >= 1
  *** QED
consP x t@(Single a) =
    consT x t + phi (cons x t) - phi t
-- Use definition of cons/T _ (Single a)
  === 1 + phi (Deep (One x) EmptyT (One a))
        - phi (Single a)
-- Use definition of phi
  === 1 + dang (One x) + phi EmptyT
        + dang (One a) - 0
-- Use definition of phi and dang
  === 1 + 1 + 0 + 1 - 0
-- 3 must be <= c, so c >= 3
  *** QED
\end{lstlisting}

We omit the proof cases for one to three digits, as they work exactly as the other two base cases. The constant however will not change in the other base cases, so $3$ really is our upper bound to use here. The interesting case is when there are four digits. In this case we need to use induction to prove our goal:

\begin{lstlisting}
consP x t@(Deep (Four a b c d) m sf) =
    consT x t + phi (cons x t) - phi t
-- Use definition of consT
  === (1 + consT (Node3 b c d) m)
  + phi (cons x t) - phi t
-- Use definition of cons
  === (1 + consT (Node3 b c d) m)
  + phi (Deep (Two x a) (cons (Node3 b c d) m) sf)
  - phi t
-- Use definition of phi
  === (1 + consT (Node3 b c d) m)
  + dang (Two x a) + phi (cons (Node3 b c d) m)
  + dang sf
  - (dang (Four a b c d) + phi m + dang sf)
-- Remove parenthesis
  === 1 + consT (Node3 b c d) m
  + dang (Two x a) + phi (cons (Node3 b c d) m)
  + dang sf
  - dang (Four a b c d) - phi m - dang sf
-- dang sf cancels, use definition of dang
  === 1 + consT (Node3 b c d) m
  + 0 + phi (cons (Node3 b c d) m) - 1 - phi m
-- 1s cancel
  === consT (Node3 b c d) m
  + phi (cons (Node3 b c d) m) - phi m
-- Use induction
  ? consP (Node3 b c d) m
  *** QED
\end{lstlisting}

\subsubsection{Append}\label{sec:append}

Similar to the \texttt{push} operation on the stack, even though \texttt{append} has the same amortized complexity as its normal complexity, we need to show that it never increases the potential more than a logarithmic amount. Contrary to the simplified definition of Claessen~\cite{fingertrees_new}, the \textit{containers} package uses two sets of mutually recursive functions to implement \texttt{append}. One set of functions called \texttt{appendTreeX} handles the base cases where one of the two trees is either empty, or consists of a singular element by using \texttt{cons} and \texttt{snoc}. In the recursive case it calls the \texttt{addDigitsX} functions that group arguments as \texttt{Node}s and in turn call the \texttt{appendTreeX} functions. Because the \texttt{addDigitsX} functions do not pattern match on their tree arguments at all, LiquidHaskell can not find a termination metric automatically. For this, we need to track the arguments across functions calls because \texttt{appendTreeX} has a strictly decreasing argument. We can do so by adding a ghost parameter to order the functions for LiquidHaskell. This order is defined to be smaller for all \texttt{appendTreeX} functions than the \texttt{addDigitsX} functions. Additionally we need to set the case expansion to a high number such that all pattern matches are evaluated, else the termination proof would fail. The following snippet provides an example with most cases and function omitted for readability.

\begin{lstlisting}
{-@ LIQUID --max-case-expand=1000 @-}

{-@ appendTree1 :: a:FingerTree (Node a) ->
  Node a -> FingerTree (Node a) ->
  FingerTree (Node a) / [size a, 1] @-}
{-@ reflect appendTree1 @-}
appendTree1 :: FingerTree (Node a) -> Node a ->
    FingerTree (Node a) -> FingerTree (Node a)
appendTree1 EmptyT !a xs = a `consTree` xs
appendTree1 xs !a EmptyT = xs `snocTree` a
appendTree1 (Single x) !a xs =
    x `consTree` a `consTree` xs
appendTree1 xs !a (Single x) =
    xs `snocTree` a `snocTree` x
appendTree1 (Deep pr1 m1 sf1) a (Deep pr2 m2 sf2) =
    Deep pr1 (addDigits1 m1 sf1 a pr2 m2) sf2

{-@ addDigits1 :: a:FingerTree (Node (Node a)) ->
  Digit (Node a) -> Node a -> Digit (Node a) ->
  FingerTree (Node (Node a)) ->
  FingerTree (Node (Node a)) / [size a, 5] @-}
{-@ reflect addDigits1 @-}
addDigits1 :: FingerTree (Node (Node a)) ->
  Digit (Node a) -> Node a -> Digit (Node a) ->
  FingerTree (Node (Node a)) ->
  FingerTree (Node (Node a))
addDigits1 m1 (One a) b (One c) m2 =
  appendTree1 m1 (Node3 a b c) m2
addDigits1 m1 (One a) b (Two c d) m2 =
  appendTree2 m1 (Node2 a b) (Node2 c d) m2
-- further cases omitted
\end{lstlisting}

This case expansion results in an explosion of constraints for the SMT solver to check. While the whole proof for \texttt{cons} including the defined functions generates fewer than 100 constraints, this termination proof generates over 3000. This leads to extremely high compile times which is further complicated by the need to define a timing function. The function needs to follow this mutually recursive structure and generates an additional 8000 constraints, such that loading the file takes several minutes. Another problem is the logarithmic runtime itself. SMT solvers are great with linear arithmetic, but are significantly slower when tasked with multiplication. Division is usually left as an unevaluated function. While it is possible to transform the proof goal to convert logarithms into powers, it still requires assumptions about basic arithmetic rules on the logic level.

For those reasons, using LiquidHaskell to proof the complexity with the physicist's method is not feasible (see section~\ref{sec:limitations} for a discussion of those limitations). Claessen~\cite{fingertrees_new} does not formally prove this in his functional pearl as well, but provides a different reasoning to why \texttt{append} is logarithmic. The potential only depends on the spine of the tree, so a new layer in the tree only adds at most a constant amount to the potential ($2$ if both digits are dangerous) making it grow linearly. Because the nodes in the digits always at least double (in the case of two \texttt{Node2}) the overall size of the tree grows exponentially. Thus the potential grows logarithmic in terms of the size of the tree.

\section{Limitations of our approach}\label{sec:limitations}

\subsection{Definition of the timing function}

The analysis could lead to wrong results is if the timing function does not directly mirror the original function. When the implementation changes and the timing function does not, it might be possible for the proofs to still pass despite being invalid. While this is unlikely, given how verbose proofs with LiquidHaskell are, it is not negligible.

If it is possible to change the implementation, it would be possible to factor out the recursion and share it between the implementation and the timing function. This however would clutter the implementation and most likely decrease performance as it is often the case with recursion schemes.

A better solution is to derive the timing function using meta programming in the form of \textit{TemplateHaskell}. Given the name of the function in the implementation, TemplateHaskell would allow to create a new function with a configurable suffix. This can be done by checking if the right hand sides of the original function contain recursive calls to the given name. If not, replace the right hand side with $1$, otherwise change the recursion to use the name with the suffix and add $1$ at the end. The resulting timing function is then be spliced back into the source code:

\begin{lstlisting}
cons :: a -> FingerTree a -> FingerTree a
cons a EmptyT = Single a
cons a (Deep (Four b c d e) m sf) =
    Deep (Two a b) (cons (Node3 c d e) m) sf
-- other cases omitted

$(mkTimingFn "T" ''cons)
-- generates this; variable names omitted:
-- consT :: a -> FingerTree a -> Int
-- consT _ EmptyT = 1
-- consT _ (Deep (Four _ c d e) m _) =
--    consT (Node3 c d e) m + 1
-- other cases omitted
\end{lstlisting}

\subsection{LiquidHaskell for verification}

As shown in earlier sections, at the time of writing LiquidHaskell has some limitations with regards to proving more complex functions. First and foremost the support for type classes and Haskell \texttt{newtype}s is insufficient. Currently, LiquidHaskell only supports measure classes, which need to be defined by recursion over a single data type similar to a normal measure. For this reason it is not possible to use them for the size parameter in finger trees (see section~\ref{sec:fingertree}). There are several open tickets concerning the type class support on the LiquidHaskell bug tracker, so we expect this to improve in the future.

Another road block was the combinatorial explosion when generating the constraints for the termination of \texttt{append} (see section~\ref{sec:append}). It might be possible to improve the constraint generation and thus create less work for the SMT solver. LiquidHaskell also provides an option to disable the termination checker, but with this option the compile time surprisingly did not change. Similarly, several other options that are supposed to improve performance like \texttt{{-}{-}diff}, which enables incremental type checking or \texttt{{-}{-}fast}, which disables inference of polymorphic refinement, did either not affect performance at all or (in the case of \texttt{{-}{-}fast}) made it even worse.

\section{Conclusion}\label{sec:conclusion}

Despite the current limitations of LiquidHaskell preventing more complex functions from being formally proven, it drastically improves the readability of proofs compared to other theorem provers. The explicit equational reasoning of LiquidHaskell makes following the proof a lot easier than in tactic based provers like Coq and Lean. While Isabelle/HOL also features equational reasoning with Isar~\cite{isar}, it does not support polymorphic recursion which makes the definition of our second example a lot harder.

While at the moment amortized complexity of \O{1} seems to be the most feasable to prove, we expect the tooling around LiquidHaskell to improve a lot in the future, be it by incremental changes like better support for type classes, or by more complicated means like a deeper embedding into a future dependently typed Haskell core. The recent improvements in the usability of LiquidHaskell --- that were achieved by turning the previously standalone executable into a GHC plugin~\cite{lh_plugin} --- are already a big step into this future by drastically reducing the barrier of entry.

%%
%% The acknowledgments section is defined using the "acks" environment
%% (and NOT an unnumbered section). This ensures the proper
%% identification of the section in the article metadata, and the
%% consistent spelling of the heading.
%\begin{acks}
%\end{acks}

%%
%% The next two lines define the bibliography style to be used, and
%% the bibliography file.
\bibliographystyle{ACM-Reference-Format}
\bibliography{references}

%%
%% If your work has an appendix, this is the place to put it.
\appendix

\section{Full proof code for the stack}

\begin{lstlisting}
import Language.Haskell.Liquid.ProofCombinators

data Stack a = Empty | Elem a (Stack a)

{-@ reflect push @-}
push :: a -> Stack a -> Stack a
push = Elem

{-@ multipop :: Nat -> Stack a -> Stack a @-}
multipop :: Int -> Stack a -> Stack a
multipop _ Empty = Empty
multipop 0 s = s
multipop n (Elem _ s) = multipop (n - 1) s

{-@ reflect phi @-}
{-@ phi :: Stack a -> Nat @-}
phi :: Stack a -> Int
phi Empty = 0
phi (Elem _ s) = 1 + phi s

{-@ reflect pushT @-}
{-@ pushT :: a -> Stack a -> Nat @-}
pushT :: a -> Stack a -> Int
pushT _ _ = 1

{-@ reflect multipopT @-}
{-@ multipopT :: Nat -> Stack a -> Nat @-}
multipopT :: Int -> Stack a -> Int
multipopT _ Empty = 1
multipopT 0 _ = 1
multipopT n (Elem _ s) = 1 + multipopT (n - 1) s


{-@ pushP :: x:a -> s:Stack a ->
  { pushT x s + phi (push x s) - phi s <= 2 }
@-}
pushP :: a -> Stack a -> Proof
pushP x s = pushT x s + phi (push x s) - phi s
-- Use definition of push and pushT
  === 1 + phi (Elem x s) - phi s
-- Use definition of phi (Elem _ _)
  === 1 + (1 + phi s) - phi s
-- phi s gets canceled, 1 + 1 <= 2
  *** QED

{-@ multipopP :: k:Nat -> s:Stack a ->
  { multipopT k s + phi (multipop k s)
      - phi s <= 2 }
@-}
multipopP :: Int -> Stack a -> Proof
multipopP 0 s = multipopT 0 s
    + phi (multipop 0 s) - phi s
-- Use definition of multipop/T 0 _
  === 1 + phi s - phi s
-- phi s cancels, 1 <= 2
  *** QED
multipopP k Empty = multipopT k Empty
    + phi (multipop k Empty) - phi Empty
-- Use definition of multipop/T _ Empty
  === 1 + phi Empty - phi Empty
-- phi Empty cancels, 1 <= 2
  *** QED
multipopP k xs@(Elem x s) = multipopT k xs
    + phi (multipop k xs) - phi xs
-- Use definition of multipopT for recusive case
  === (1 + multipopT (k - 1) s)
    + phi (multipop k xs) - phi xs
-- Use definition of multipop for recusive case
  === (1 + multipopT (k - 1) s)
    + phi (multipop (k - 1) s) - phi xs
-- Use definition of phi (Elem _ _)
  === (1 + multipopT (k - 1) s)
    + phi (multipop (k - 1) s) - (1 + phi s)
-- Remove parenthesis
  === 1 + multipopT (k - 1) s
    + phi (multipop (k - 1) s) - 1 - phi s
-- 1s cancel out
  === multipopT (k - 1) s
    + phi (multipop (k - 1) s) - phi s
-- use induction
  ? multipopP (k - 1) s
  *** QED
\end{lstlisting}

\section{Full proof code of the finger tree}\label{ap:fingertree}

\begin{lstlisting}
data FingerTree a = Empty
  | Single a
  | Deep (Digit a) (FingerTree (Node a)) (Digit a)

data Node a = Node2 a a | Node3 a a a

data Digit a = One a
  | Two a a
  | Three a a a
  | Four a a a a

{-@ reflect cons @-}
cons :: a -> FingerTree a -> FingerTree a
cons a EmptyT = Single a
cons a (Single b) =
  Deep (One a) EmptyT (One b)
cons a (Deep (One b) m sf) =
  Deep (Two a b) m sf
cons a (Deep (Two b c) m sf) =
  Deep (Three a b c) m sf
cons a (Deep (Three b c d) m sf) =
  Deep (Four a b c d) m sf
cons a (Deep (Four b c d e) m sf) =
  Deep (Two a b) (cons (Node3 c d e) m) sf

{-@ reflect snoc @-}
snoc :: FingerTree a -> a -> FingerTree a
snoc EmptyT x = Single x
snoc (Single a) x =
  Deep (One a) EmptyT (One x)
snoc (Deep pr m (One a)) x =
  Deep pr m (Two a x)
snoc (Deep pr m (Two a b)) x =
  Deep pr m (Three a b x)
snoc (Deep pr m (Three a b c)) x =
  Deep pr m (Four a b c x)
snoc (Deep pr m (Four a b c d)) x =
  Deep pr (snoc m (Node3 a b c)) (Two d x)

{-@ reflect dang @-}
{-@ dang :: Digit a -> Nat @-}
dang :: Digit a -> Int
dang One{} = 1
dang Two{} = 0
dang Three{} = 0
dang Four{} = 1

{-@ reflect phi @-}
{-@ phi :: FingerTree a -> Nat @-}
phi :: FingerTree a -> Int
phi EmptyT = 0
phi Single{} = 0
phi (Deep pr m sf) =
  dang pr + phi m + dang sf

{-@ reflect consT @-}
{-@ consT :: a -> FingerTree a -> Nat @-}
consT :: a -> FingerTree a -> Int
consT _ EmptyT = 1
consT _ Single{} = 1
consT _ (Deep One{} _ _) = 1
consT _ (Deep Two{} _ _) = 1
consT _ (Deep Three{} _ _) = 1
consT _ (Deep (Four _ c d e) m _) =
  1 + consT (Node3 c d e) m

{-@ reflect snocT @-}
{-@ snocT :: FingerTree a -> a -> Nat @-}
snocT :: FingerTree a -> a -> Int
snocT EmptyT _ = 1
snocT Single{} _ = 1
snocT (Deep _ _ One{}) _ = 1
snocT (Deep _ _ Two{}) _ = 1
snocT (Deep _ _ Three{}) _ = 1
snocT (Deep _ m (Four a b c _)) _ =
    1 + snocT m (Node3 a b c)

{-@ consP :: x:a -> t:FingerTree a ->
  { consT x t + phi (cons x t) - phi t <= 3 }
@-}
consP :: a -> FingerTree a -> Proof
consP x t@EmptyT =
    consT x t + phi (cons x t) - phi t
-- Use definition of cons/T _ EmptyT
  === 1 + phi (Single x) - phi EmptyT
-- Use definition of phi
  === 1 + 0 - 0 -- trivial, 1 <= 3
  *** QED
consP x t@(Single a) =
    consT x t + phi (cons x t) - phi t
-- Use definition of cons/T _ (Single a)
  === 1 + phi (Deep (One x) EmptyT (One a))
        - phi (Single a)
-- Use definition of phi
  === 1 + dang (One x) + phi EmptyT
        + dang (One a) - 0
-- Use definition of phi and dang
  === 1 + 1 + 0 + 1 - 0 -- trivial, 3 <= 3
  *** QED
consP x t@(Deep (One a) m sf) =
    consT x t + phi (cons x t) - phi t
-- Use definition of cons/T
  === 1 + phi (Deep (Two x a) m sf) - phi t
-- Use definition of phi
  === 1 + dang (Two x a) + phi m + dang sf
    - (dang (One a) + phi m + dang sf)
-- Remove parenthesis, use definition of dang
  === 1 + 0 + phi m + dang sf
    - 1 - phi m - dang sf
-- everything cancels
  === 0 -- trivial, 0 <= 3
  *** QED
consP x t@(Deep (Two a b) m sf) =
    consT x t + phi (cons x t) - phi t
-- Use definition of cons/T
  === 1 + phi (Deep (Three x a b) m sf) - phi t
-- Use definition of phi
  === 1 + dang (Three x a b) + phi m + dang sf
    - (dang (Two a b) + phi m + dang sf)
-- Remove parenthesis, use definition of dang
  === 1 + 0 + phi m + dang sf
    - 0 - phi m - dang sf
-- phi and dang cancels
  === 1 -- trivial, 1 <= 3
  *** QED
consP x t@(Deep (Three a b c) m sf) =
    consT x t + phi (cons x t) - phi t
-- Use definition of cons/T
  === 1 + phi (Deep (Four x a b c) m sf) - phi t
-- Use definition of phi
  === 1 + dang (Four x a b c) + phi m + dang sf
    - (dang (Three a b c) + phi m + dang sf)
-- Remove parenthesis, use definition of dang
  === 1 + 1 + phi m + dang sf
    - 0 - phi m - dang sf
-- phi and dang cancels
  === 2 -- trivial, 2 <= 3
  *** QED
consP x t@(Deep (Four a b c d) m sf) =
    consT x t + phi (cons x t) - phi t
-- Use definition of consT
  === (1 + consT (Node3 b c d) m)
  + phi (cons x t) - phi t
-- Use definition of cons
  === (1 + consT (Node3 b c d) m)
  + phi (Deep (Two x a) (cons (Node3 b c d) m) sf)
  - phi t
-- Use definition of phi
  === (1 + consT (Node3 b c d) m)
  + dang (Two x a) + phi (cons (Node3 b c d) m)
  + dang sf
  - (dang (Four a b c d) + phi m + dang sf)
-- Remove parenthesis
  === 1 + consT (Node3 b c d) m
  + dang (Two x a) + phi (cons (Node3 b c d) m)
  + dang sf
  - dang (Four a b c d) - phi m - dang sf
-- dang sf cancels, use definition of dang
  === 1 + consT (Node3 b c d) m
  + 0 + phi (cons (Node3 b c d) m) - 1 - phi m
-- 1s cancel
  === consT (Node3 b c d) m
  + phi (cons (Node3 b c d) m) - phi m
-- Use induction
  ? consP (Node3 b c d) m
  *** QED

{-@ snocP :: t:FingerTree a -> x:a ->
  { snocT t x + phi (snoc t x) - phi t <= 3 }
@-}
snocP :: a -> FingerTree a -> Proof
snocP t@EmptyT x =
    snocT t x + phi (snoc t x) - phi t
-- Use definition of snoc/T EmptyT _
  === 1 + phi (Single x) - phi EmptyT
-- Use definition of phi
  === 1 + 0 - 0 -- trivial, 1 <= 3
  *** QED
snocP t@(Single a) x =
    snocT t x + phi (snoc t x) - phi t
-- Use definition of snoc/T (Single a) _
  === 1 + phi (Deep (One a) EmptyT (One x))
        - phi (Single a)
-- Use definition of phi
  === 1 + dang (One a) + phi EmptyT
        + dang (One x) - 0
-- Use definition of phi and dang
  === 1 + 1 + 0 + 1 - 0 -- trivial, 3 <= 3
  *** QED
snocP t@(Deep pr m (One a)) x =
    snocT t x + phi (snoc t x) - phi t
-- Use definition of snoc/T
  === 1 + phi (Deep pr m (Two a x)) - phi t
-- Use definition of phi
  === 1 + dang pr + phi m + dang (Two a x)
    - (dang pr + phi m + dang (One a))
-- Remove parenthesis, use definition of dang
  === 1 + dang pr + phi m + 0
    - dang pr - phi m - 1
-- everything cancels
  === 0 -- trivial, 0 <= 3
  *** QED
snocP t@(Deep pr m (Two a b)) x =
    snocT t x + phi (snoc t x) - phi t
-- Use definition of snoc/T
  === 1 + phi (Deep pr m (Three a b x)) - phi t
-- Use definition of phi
  === 1 + dang pr + phi m + dang (Three a b x)
    - (dang pr + phi m + dang (Two a b))
-- Remove parenthesis, use definition of dang
  === 1 + dang pr + phi m + 0
    - dang pr - phi m - 0
-- phi and dang cancels
  === 1 -- trivial, 1 <= 3
  *** QED
snocP t@(Deep pr m (Three a b c)) x =
    snocT t x + phi (snoc t x) - phi t
-- Use definition of snoc/T
  === 1 + phi (Deep pr m (Four a b c x)) - phi t
-- Use definition of phi
  === 1 + dang pr + phi m + dang (Four a b c x)
    - (dang pr + phi m + dang (Three a b c))
-- Remove parenthesis, use definition of dang
  === 1 + dang pr + phi m + 1
    - dang pr - phi m - 0
-- phi and dang cancels
  === 2 -- trivial, 2 <= 3
  *** QED
snocP t@(Deep pr m (Four a b c d)) x =
    snocT t x + phi (snoc t x) - phi t
-- Use definition of snocT
  === (1 + snocT m (Node3 a b c))
  + phi (snoc t x) - phi t
-- Use definition of snoc
  === (1 + snocT m (Node3 a b c))
  + phi (Deep pr (snoc m (Node3 a b c)) (Two x a))
  - phi t
-- Use definition of phi
  === (1 + snocT m (Node3 a b c))
  + dang pr + phi (snoc m (Node3 a b c))
  + dang (Two x a)
  - (dang pr + phi m + dang (Four a b c d))
-- Remove parenthesis
  === 1 + snocT m (Node3 a b c)
  + dang pr + phi (snoc m (Node3 a b c))
  + dang (Two x a)
  - dang pr - phi m - dang (Four a b c d))
-- dang pr cancels, use definition of dang
  === 1 + snocT m (Node3 a b c)
  + phi (snoc m (Node3 a b c)) + 0 - phi m - 1
-- 1s cancel
  === snocT m (Node3 a b c)
  + phi (snoc m (Node3 a b c)) - phi m
-- Use induction
  ? snocP m (Node3 a b c)
  *** QED


\end{lstlisting}

\end{document}
\endinput
